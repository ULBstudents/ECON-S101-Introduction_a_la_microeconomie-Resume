\section{Séance 8 : Échanges Internationaux}




\subsection{Avantages comparatifs}



Les pays 1 et 2 produisent les biens X et Y.\\
Les chiffres représentent les coûts de production (en heures de travail ou en salaire).
\begin{center}
	\begin{tabular}{|c|c|c|}
		\hline
		       & Pays 1 & Pays 2\\\hline
		Bien X & 10     & 12    \\\hline
		Bien Y & 5      & 4    \\\hline
	\end{tabular}
\end{center}

\begin{description}
	\item Le \textbf{pays 1} a un avantage absolu dans la production du \textbf{bien X} ;
	\item Le \textbf{pays 2} a un avantage absolu dans la production du \textbf{bien Y} ;
	\item Le \textbf{$TMT_{de\ X_1\ pour\ un\ Y_1} = \dfrac{5}{10} = \dfrac{1}{2}$ } ;
	\item Le \textbf{$TMT_{de\ X_2\ pour\ un\ Y_2} = \dfrac{4}{12} = \dfrac{1}{3}$ } ;
	\item Le \textbf{$TMT_{de\ Y_1\ pour\ un\ X_1} = \dfrac{10}{5} = 2$ } ;
	\item Le \textbf{$TMT_{de\ Y_2\ pour\ un\ X_2} = \dfrac{12}{4} = 3$ }.
\end{description}